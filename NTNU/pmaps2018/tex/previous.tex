\section{Previous work}
\begin{frame}
		\frametitle{Previous articles}
	\begin{columns}
		\begin{column}{0.5\textwidth}
			\begin{itemize}
				\item<1-> Governor dynamics were identified using the ARX model structure
				\item<2-> Governor dynamics were identified using time domain vector fitting
				\item<3-> There are also other papers in the literature using other methods for online identification, however, mostly relying on data from disturbance recordings.
			\end{itemize}
		\end{column}
		\begin{column}{0.5\textwidth}
			\begin{figure}
				\includegraphics<1>[width=0.7\textwidth]{./pictures/thuc_bode}
				\includegraphics<2->[width=0.7\textwidth]{./pictures/bode.tikz}
			\end{figure}
		\end{column}
	\end{columns}
\end{frame}
\begin{frame}
		\frametitle{Question leading to this specific work}
	\begin{columns}
		\begin{column}{0.5\textwidth}
			\begin{itemize}
				\item<1-> Why do we get different results?
				\item<2-> The signals we use are corrupted by noise.
				\item<3-> Using system identification techniques we can estimate the variance of the covariance matrix of the parameter vector
				\item<4> However, first we have to prove that we will get consistent results.
			\end{itemize}
		\end{column}
		\begin{column}{0.5\textwidth}
				\includegraphics<1>[width=0.7\textwidth]{./pictures/bode.tikz}
				\includegraphics<2>{./pictures/v_block.tikz}
				\begin{equation*}\onslide<3>
						a+b
				\end{equation*}
		\end{column}
	\end{columns}
\end{frame}
