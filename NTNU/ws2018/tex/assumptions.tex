\section{Assumption that $\Delta \omega_1[n] \approx \Delta \omega_3[n]$}
\begin{frame}{\secname}
	\begin{itemize}
		\item In reality $\Delta \omega_1[n] \approx \Delta \omega_3[n]$ is only valid for a certain frequency range.
		\item We need an expression for the angular velocity at bus $3$.
		\item The two standard options would be:
			\begin{itemize}
				\item The time derivative of the voltage angle at the bus.
				\item The centre of inertia equation.
			\end{itemize}
		\item We will instead use the frequency divider(FD) formula (Milano 2017).
	\end{itemize}
\end{frame}
\begin{frame}{\secname}
	\framesubtitle{Derivation of the FD formula}
	\begin{itemize}
		\item Start by the DC power flow assumption assuming the load changes to be negligible.
		\begin{equation}
			\begin{bmatrix}
				\boldsymbol{\Delta P}_{e} \\ \boldsymbol{0}
			\end{bmatrix}
			=
			\begin{bmatrix}
				\boldsymbol{B}_{11} & \boldsymbol{B}_{12} \\
				\boldsymbol{B}_{21} & \boldsymbol{B}_{22}
			\end{bmatrix}
			\begin{bmatrix}
				\boldsymbol{\Delta \delta}_e \\
				\boldsymbol{\Delta \delta}_l
			\end{bmatrix}
		\end{equation}
		\item Then we rearrange
		\begin{equation}\label{eq:theta_l}
			\boldsymbol{\Delta \delta}_l = -\boldsymbol{B}_{22}^{-1}\boldsymbol{B}_{21}\boldsymbol{\Delta \delta}_e
		\end{equation}
		\item We now take the time derivative to obtain.
		\begin{equation}\label{eq:theta_l}
			\boldsymbol{\Delta \omega}_l = -\boldsymbol{B}_{22}^{-1}\boldsymbol{B}_{21}\boldsymbol{\Delta \omega}_e
		\end{equation}
	\end{itemize}
\end{frame}
\begin{frame}{\secname}
	\framesubtitle{FD example}
\begin{equation}
		\boldsymbol{B}=
		\left[
		\begin{array}{cc:ccc}
				b'_{d1} & 0 & -b'_{d1} & 0 & 0  \\
				0 & b'_{d2} & 0 & -b'_{d2} & 0 \\
			\hdashline
			-b'_{d1} & 0 & b'_{d1}+b_1 & 0 & -b_1 \\
			0 & -b'_{d2} & 0 & b'_{d2}+b_2 & -b_2 \\
			0 & 0 & -b_{1} & -b_{2} & b_{1}+b_2 \\
		\end{array}
		\right]
\end{equation}

\begin{equation}
		\Delta \omega_3 = \frac{(b_1b_{d2}+b_1b_2+b_2b_{d2})b_{d1}}{|\boldsymbol{B}_{22}|}\Delta\omega_1 +\frac{b_1b_2b_{d2}}{|\boldsymbol{B}_{22}|}\Delta\omega_2
\end{equation}

\begin{equation}
		\Delta \omega_5 = \frac{b_1b'_{d1}(b'_{d2}+b_2)}{|\boldsymbol{B}_{22}|}\Delta \omega_1 + \frac{b_2b'_{d2}(b'_{d1}+b_1)}{|\boldsymbol{B}_{22}|}\Delta \omega_2
\end{equation}

\begin{equation}
		\Delta \omega_{COI} = \frac{1}{\mathcal{H}_1+\mathcal{H}_2}(\mathcal{H}_1\Delta \omega_1 + \mathcal{H}_2\Delta \omega_2)
\end{equation}

\end{frame}

\begin{frame}{\secname}
	\framesubtitle{FD example}
	\begin{figure}
		\includegraphics[width=0.8\textwidth]{./pictures/frequencies.tikz}
	\end{figure}
\end{frame}
\begin{frame}{\secname}
	\framesubtitle{Assumptions further work}
	\begin{itemize}
		\item<1-> Continue with the dc power flow assumption,
		\item<2-> however, remove the FD assumption of $\Delta P_l=0$.
	\end{itemize}
\end{frame}
