\section{Validation of the approach}
\begin{frame}
	\frametitle{System identification basic}
	\begin{columns}
		\begin{column}{0.5\textwidth}
			\begin{itemize}
				\item Assume that a data set $Z^N = \{u[n],y[n]|n=1\ldots N\}$ has been collected.
				\item The dataset $Z^N$ is assumed generated by
					\begin{equation}
						\mathcal{S}: y[n] = G_1(z,\theta_1)u[n] + H_1(z,\theta_1)e[n]
					\end{equation}
				\item Using the data set $Z^N$ we want to find the parameter vector $\theta^N$ minimizing
\begin{equation}\label{eq:pred}
		\hat{\theta}_N = \argmin_{\theta} \frac{1}{N}\sum_{n=1}^N \epsilon^2(n,\theta)
\end{equation}
			\end{itemize}
		\end{column}
		\begin{column}{0.5\textwidth}
			\begin{figure}
				\includegraphics{./pictures/v_block.tikz}
			\end{figure}
		\end{column}
	\end{columns}
\end{frame}
\begin{frame}
	\frametitle{Modeling used for the validation}
	\begin{columns}
		\begin{column}{0.3\textwidth}
			\begin{itemize}[<+->]
				\item The system we want to identify
				\item We use a small power system
				\item We use a dc power flow 
				\item This results in the following block diagram
			\end{itemize}
		\end{column}
		\begin{column}{0.7\textwidth}
				\includegraphics<1>{./pictures/sys.tikz}
				\includegraphics<2>[width=0.8\textwidth]{./pictures/sld.tikz}
				\includegraphics<3>{./pictures/DC.tikz}
				\includegraphics<4>{./pictures/block.tikz}
		\end{column}
	\end{columns}
\end{frame}

\begin{frame}
	\frametitle{Conclusion from the identification analysis}
	\begin{columns}
		\begin{column}{0.3\textwidth}
			\begin{itemize}[<+->]
					\item We can identify a consistent estimate of $G_1(s)$
						\begin{itemize}
							\item If $v_{l5}(s)$ excites the system sufficiently,
							\item and there is a delay in either $G_1(s)$ of the transfer function from $\Delta \omega_1(s)$ to $\Delta P_{e1}(s)$.
						\end{itemize}
			\end{itemize}
		\end{column}
		\begin{column}{0.7\textwidth}
				\includegraphics{./pictures/block.tikz}
		\end{column}
	\end{columns}
\end{frame}

