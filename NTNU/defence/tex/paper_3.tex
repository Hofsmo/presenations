\section{Theoretical validation Paper III}
\begin{frame}
		\frametitle{Background}
	\begin{columns}
		\begin{column}{0.5\textwidth}
			\begin{itemize}
				\item<1-> Why do we get different results?
				\item<2-> The signals we use are corrupted by noise.
				\item<3-> Before we can talk about the different results, we need to validate our approach.
			\end{itemize}
		\end{column}
		\begin{column}{0.5\textwidth}
				\begin{figure}
					\includegraphics<1>[width=0.7\textwidth]{./pictures/bode.tikz}
					\includegraphics<2>{./pictures/v_block.tikz}
					\includegraphics<3>{./pictures/identifiability.tikz}
				\end{figure}
		\end{column}
	\end{columns}
\end{frame}
\begin{frame}
	\frametitle{System identification basic}
	\begin{columns}
		\begin{column}{0.5\textwidth}
			\begin{itemize}
				\item Assume that a data set $Z^N = \{u[n],y[n]|n=1\ldots N\}$ has been collected.
				\item The dataset $Z^N$ is assumed generated by
					\begin{equation}
						\mathcal{S}: y[n] = G_1(z,\theta_1)u[n] + H_1(z,\theta_1)e[n]
					\end{equation}
				\item Using the data set $Z^N$ we want to find the parameter vector $\theta^N$ minimizing
\begin{equation}\label{eq:pred}
		\hat{\theta}_N = \argmin_{\theta} \frac{1}{N}\sum_{n=1}^N[ H_1^{-1}(z,\theta)(y[n]-G_1(z,\theta)u[n])]^2
\end{equation}
			\end{itemize}
		\end{column}
		\begin{column}{0.5\textwidth}
			\begin{figure}
				\includegraphics{./pictures/v_block.tikz}
			\end{figure}
		\end{column}
	\end{columns}
\end{frame}
\begin{frame}
	\frametitle{Modeling used for the validation}
	\begin{columns}
		\begin{column}{0.3\textwidth}
			\begin{itemize}[<+->]
				\item The system we are identifying
					\begin{itemize}
						\item Not $G_p(s)$.
						\item Opposite $u$ and $y$ from Paper I.
					\end{itemize}
				\item We use a small power system
				\item We use a dc power flow 
				\item This results in the following block diagram
			\end{itemize}
		\end{column}
		\begin{column}{0.7\textwidth}
				\begin{equation}\only<1>{G_1(s) = \frac{G_p(s)}{1+G_p(s)G_J(s)}}\end{equation}
				\includegraphics<1>{./pictures/sys.tikz}
				\includegraphics<2>[width=0.8\textwidth]{./pictures/sld.tikz}
				\includegraphics<3>{./pictures/DC.tikz}
				\includegraphics<4>{./pictures/block_test_sys.tikz}
		\end{column}
	\end{columns}
\end{frame}
\begin{frame}
	\frametitle{Results from the theoretical validation}
	\begin{itemize}
		\item<1-> A consistent estimate of the closed loop transfer function of the turbine and electromechanical dynamics can be obtained by using:
			\begin{itemize}
				\item<2-> Measured PMU frequency as the output $u[n]$
				\item<3-> Measured PMU power as the input $y[n]$
			\end{itemize}
		\item<4-> The proof was done with the following assumptions.
			\begin{itemize}
				\item The system is excited by a load acting as a filtered white noise process
				\item The measurement error of the electrical power is negligible.
				\item The measured frequency is a good estimate of the generator speed.
			\end{itemize}
	\end{itemize}
\end{frame}
\begin{frame}
	\frametitle{Model obtained using PMU data}
	\begin{figure}
		\includegraphics[width=0.8\textwidth]{./pictures/PMU_bode.tikz}
	\end{figure}
\end{frame}
\begin{frame}
	\frametitle{Whiteness test on model identified using PMU data}
	\begin{figure}
		\includegraphics[width=0.7\textwidth]{./pictures/PMU_resid.tikz}
	\end{figure}
\end{frame}
\begin{frame}
	\frametitle{Main contributions}
	\begin{itemize}
		\item To show that the transfer function one is identifying using PMUs is $G_1(s)$.
		\item To prove under which conditions a consistent estimate of $G_1(s)$ is possible.
		\item To demonstrate the theory for identification of $G_1(s)$ on real datasets.
	\end{itemize}
\end{frame}
