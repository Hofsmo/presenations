\section{Theoretical validation}
\begin{frame}
	\frametitle{Motivation}
	\begin{itemize}
		\item To validate the PMU method analytically.
	\end{itemize}
	\includegraphics{./pictures/genTrafo.tikz}
\end{frame}
\begin{frame}
	\frametitle{Modelling used for the validation}
	\begin{columns}
		\begin{column}{0.3\textwidth}
			\begin{itemize}
				\item The system we want to identify is $G_1(s)$.
				\item The external perturbation to the system is $v_l(s)$.
			\end{itemize}
		\end{column}
		\begin{column}{0.7\textwidth}
			\includegraphics{./pictures/block_test_sys.tikz}
		\end{column}
	\end{columns}
\end{frame}
\begin{frame}
	\frametitle{What can we identify using a PMU}
	\begin{itemize}
		\item $\Delta \omega(s)$ is related to $\Delta P_e(s)$ by:
	\begin{equation*}
		\Delta \omega(s) = \frac{G_{J}}{1+G_p(s)G_J(s)}\Delta P_e(s)
	\end{equation*}
	\item This means that we can identify:
		\begin{equation*}
			G_1(s) = \frac{G_{J}}{1+G_p(s)G_J(s)}
		\end{equation*}
	\end{itemize}
		\includegraphics{./pictures/sys_G1.tikz}
\end{frame}
\begin{frame}
	\frametitle{Assumptions for the theoretical validation}
		\begin{itemize}
			\item The system is excited by a load acting as a filtered white noise process
			\item The measurement error of the electrical power is negligible.
			\item The measured frequency is a good estimate of the generator speed.
		\end{itemize}
\end{frame}
\begin{frame}
	\frametitle{Assumptions for the theoretical validation}
	\framesubtitle{The system is excited by a load acting as a filtered white noise process}
	\includegraphics{./pictures/sld.tikz}
	\includegraphics[width=0.8\textwidth]{./pictures/aura_pmu.tikz}
\end{frame}
\begin{frame}
	\frametitle{Assumptions for the theoretical validation}
	\framesubtitle{The measurement error of the electrical power is negligible}
	\begin{columns}
		\begin{column}{0.5\textwidth}
			\begin{itemize}
				\item $u[n] = P[n]$
				\item $y[n] = f[n]$
			\end{itemize}
		\end{column}
		\begin{column}{0.5\textwidth}
			\includegraphics{./pictures/v_block.tikz}
		\end{column}
	\end{columns}
	\includegraphics{./pictures/genTrafo.tikz}
\end{frame}
\begin{frame}
	\frametitle{Assumptions for the theoretical validation}
	\framesubtitle{The measured frequency is a good estimate of the generator speed}
	\begin{columns}
		\begin{column}{0.35\textwidth}
			\begin{itemize}
				\item $f(s)$ is the fundamental frequency of the voltage.
				\item $\omega(s)$ is the rotational speed of the generator rotor.
			\end{itemize}
		\end{column}
		\begin{column}{0.65\textwidth}
				\includegraphics[width=\textwidth]{./pictures/genTrafo.tikz}
		\end{column}
	\end{columns}
	\includegraphics{./pictures/sys_G1.tikz}
\end{frame}
\begin{frame}
	\frametitle{Results from the theoretical validation}
	\begin{itemize}
		\item A consistent estimate of $G_1(s)$ can be obtained by using:
		\begin{itemize}
			\item Measured PMU frequency as the output $u[n]$
			\item Measured PMU power as the input $y[n]$
			\item If there is a delay between $\Delta \omega(s)$ and $\Delta P_e(s)$.
		\end{itemize}
		\begin{equation*}
			G_1(s) = \frac{G_{J}}{1+G_p(s)G_J(s)}
		\end{equation*}
	\end{itemize}
	\includegraphics{./pictures/sys_G1.tikz}
\end{frame}
\begin{frame}
	\frametitle{Model obtained using PMU data}
	\begin{figure}
		\includegraphics[width=0.8\textwidth]{./pictures/PMU_bode.tikz}
	\end{figure}
\end{frame}
\begin{frame}
	\frametitle{Main contributions}
	\begin{itemize}
		\item To show that the transfer function that can be identified using a PMU is $G_1(s)$.
		\item To prove under which conditions a consistent estimate of $G_1(s)$ is possible.
		\item To demonstrate the theory for identification of $G_1(s)$ on a real dataset.
	\end{itemize}
\end{frame}
