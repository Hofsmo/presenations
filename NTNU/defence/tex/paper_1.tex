\section{Initial tests using time domain vector fitting Paper I}
\begin{frame}
	\frametitle{Background}
	\begin{columns}
		\begin{column}{0.4\textwidth}
			\begin{itemize}
				\item<1->  Idea from\footnotemark[1] can the power plant dynamics be identified using PMUs
				\item<2-> Uses the same input and output measurements as in the requirements:
				\begin{itemize}
					\item Input: Power system frequency.
					\item Output: Electric power.
					\end{itemize}
			\end{itemize}
		\end{column}
		\begin{column}{0.6\textwidth}
			\begin{figure}
				\includegraphics<1>{./pictures/genTrafo.tikz}
				\includegraphics<2>{./pictures/block.tikz}
			\end{figure}
		\end{column}
	\end{columns}
	\footnotetext[1]{\fullcite{dinh_thuc_duong_estimation_2016}}
\end{frame}
\begin{frame}
	\frametitle{Methodology}
	\begin{columns}
		\begin{column}{0.4\textwidth}
			\begin{itemize}
				\item Collect several datasets from PMUs.
				\item Calculate power and frequency from the measurements.
				\item Identify dynamics using vector fitting.
				\item Compare models.
			\end{itemize}
		\end{column}
		\begin{column}{0.6\textwidth}
			\begin{figure}
				\includegraphics{./pictures/genTrafo.tikz}
			\end{figure}
		\end{column}
	\end{columns}
\end{frame}
\begin{frame}
	\frametitle{Vector fitting basics}
	\begin{columns}[c]
		\begin{column}{0.4\textwidth}
			\begin{itemize}
				\item<1-> Vector fitting fits a transfer function to measured input and output data
				\item<2-> It assumes the system to have the following structure.
				\item<3-> In time domain it is.
			\end{itemize}
		\end{column}
		\begin{column}{0.6\textwidth}
			\begin{itemize}
				\item[]<1->
					\begin{equation}
						Y(s) = H(s)\cdot U(s)
					\end{equation}
				\item[]<2->
					\begin{equation}
						H(s) = d + \sum^{n_p}_{i=1}\frac{r_i}{s-p_i}
					\end{equation}
				\item[]<3->
					\begin{equation}\label{eq:VFTD}
   			 			y(t) \approx \tilde{d} x(t) + \sum^{n_p}_{i=1} \tilde{r}_ix_i-\sum^{n_p}_{i=1}\tilde{k}_iy_i
					\end{equation}
					\begin{equation}\label{eq:XVFWave}
    					x_i = \int^t_0 e^{\tilde{p}_i(t-\tau)}x_i(\tau)d\tau
					\end{equation}
					\begin{equation}\label{eq:YVFWave}
    					y_i = \int^t_0 e^{\tilde{p}_{i}(t-\tau)}y_i(\tau)d\tau
					\end{equation}
			\end{itemize}
		\end{column}
	\end{columns}
\end{frame}
\begin{frame}
	\frametitle{Vector fitting basics ctd.}
	\begin{itemize}
		\item Find $\tilde{d}$, $\tilde{r}_i$ and $\tilde{k}_i$ to minimize:
	\end{itemize}
	\begin{equation}
		y(t) - (\tilde{d} x(t) + \sum^{n_p}_{i=1} \tilde{r}_ix_i-\sum^{n_p}_{i=1}\tilde{k}_iy_i)
	\end{equation}
\end{frame}
\begin{frame}[fragile]
	\frametitle{Cross validation using distant data sets}
	\includegraphics<1>[width=0.45\textwidth]{./pictures/frequencies.tikz}
	\includegraphics<1>[width=0.45\textwidth]{./pictures/powers.tikz}
	\includegraphics<2>[width=\textwidth]{./pictures/cross_val}
\end{frame}
\begin{frame}
	\frametitle{Estimated droop and bandwidth}
	\begin{figure}
		\includegraphics[width=0.6\textwidth]{./pictures/bode.tikz}
	\end{figure}
	\begin{tabular}{|c|c|c|}
		\hline
		Dataset & Droop[\%] & Bandwidth[mHz] \\ \hline
		Fall 2015 & 10 & $4.16$\\ \hline
		Spring 2016 & 8 & $2.41$\\ \hline
	\end{tabular}
\end{frame}
\begin{frame}
	\frametitle{Shortcoming with the paper}
	\begin{columns}
		\begin{column}{0.4\textwidth}
			\begin{itemize}
				\item No theoretical validation of the results.
				\item No simulation validation of the results.
			\end{itemize}
		\end{column}
		\begin{column}{0.6\textwidth}
			\begin{figure}
				\includegraphics{./pictures/sys.tikz}
			\end{figure}
		\end{column}
	\end{columns}
\end{frame}
\begin{frame}
	\frametitle{Main contributions to the research questions}
	\begin{columns}
		\begin{column}{0.4\textwidth}
			\begin{itemize}
				\item<1-> Promising results for 19 datasets.
				\item<2-> Developed code for interfacing with the PMU data.
			\end{itemize}
		\end{column}
		\begin{column}{0.6\textwidth}
			\begin{figure}
				\includegraphics<1>{./pictures/genTrafo.tikz}
			\end{figure}
		\end{column}
	\end{columns}
\end{frame}
