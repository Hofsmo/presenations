\section{Paper II}
\begin{frame}
	\frametitle{Motivation}
	\begin{columns}
		\begin{column}{0.4\textwidth}
			\begin{itemize}
				\item Explain the problem to my co-supervisor.
				\item Create a model for analysing the identifiability of hydro power plant dynamics.
			\end{itemize}
		\end{column}
		\begin{column}{0.6\textwidth}
			\begin{figure}
				\includegraphics{./pictures/sys.tikz}
			\end{figure}
		\end{column}
	\end{columns}
\end{frame}
\begin{frame}
	\frametitle{What do we need to model?}
	\begin{columns}
		\begin{column}{0.4\textwidth}
			\begin{itemize}
				\item<1-> From the PMU we get
				\begin{itemize}
					\item<2-> Power: $\Delta P_{e1}(s)$.
					\item<3-> Frequency: $\Delta f(s)$.
				\end{itemize}
				\item<4-> We need to model how $\Delta P_{e1}(s)$ and $\Delta f(s)$ is related through the power system.
				\item<5-> We also need to model the power plant consisting of $G_p(s)$ and $G_J(s)$.
			\end{itemize}
		\end{column}
		\begin{column}{0.6\textwidth}
			\begin{figure}
				\includegraphics<1>{./pictures/genTrafo.tikz}
				\includegraphics<2->{./pictures/sys.tikz}
			\end{figure}
		\end{column}
	\end{columns}
\end{frame}
\begin{frame}
	\frametitle{Power plant model}
		\begin{itemize}
			\item Model for $G_p(s)$
			\item Model for $G_J(s)$
			\begin{equation}
				G_J(s) = \frac{1}{2Hs+K_d}
			\end{equation}
		\end{itemize}
		\begin{figure}
			\includegraphics{./pictures/PID.tikz}
		\end{figure}
\end{frame}
\begin{frame}
	\frametitle{Power system model}
	\begin{columns}
		\begin{column}{0.25\textwidth}
			\begin{itemize}
				\item<1-> The frequency and power system angle is related.
				\item<2-> The angle and power is related.
				\item<3-> On matrix form.
				\item<4-> In software
			\end{itemize}
		\end{column}
		\begin{column}{0.75\textwidth}
			\begin{itemize}
				\item[]<1->
				\begin{equation}
					\Delta \theta(s) = \frac{2\pi f_s}{s}f(s)
				\end{equation}
				\item[]<2->
				\begin{equation}
					P_k \approx \sum_{m\in \Omega_k}{x^{-1}_{km}\theta_{km}}
				\end{equation}
				\item[]<3->
				\begin{equation}
					\mathbf{P}=\mathbf{Y}\mathbf{\theta}
				\end{equation}
				\includegraphics<4>{./pictures/DC.tikz}
			\end{itemize}
		\end{column}
	\end{columns}
\end{frame}
\begin{frame}
	\frametitle{Test system}
		\begin{figure}
			\includegraphics<1>[width=0.8\textwidth]{./pictures/sld.tikz}
			\includegraphics<2>{./pictures/block_test_sys.tikz}
			\caption{\only<1>{Single line diagram}\only<2>{Block diagram}}
		\end{figure}
\end{frame}
\begin{frame}
	\frametitle{Simulation Result}
	\begin{figure}
		\includegraphics[width=0.8\textwidth]{./pictures/frequency_comp.tikz}
	\end{figure}
\end{frame}
\begin{frame}
	\frametitle{Main contributions}
	\begin{itemize}
		\item Developed simple test system for analysing power plant identifiability using PMUs.
		\item Developed simple test system used in the proceeding papers for simulations.
	\end{itemize}
\end{frame}
