\section{The best way to do the identification}
\begin{frame}
	\frametitle{Motivation}
	\begin{itemize}
		\item How to best check the requirements given access to control system data.
	\end{itemize}
\end{frame}
\begin{frame}
	\frametitle{What do we want to check}
	\begin{columns}
		\begin{column}{0.6\textwidth}
			\begin{itemize}
				\item Stability requirement $|S(j\Omega)| < M_s$
				\item Performance requirement $|G_1(j\Omega)| < M_p$
				\item In the requirements this is done by identifying.
			\begin{itemize}
				\item $G_p(s)$ in open loop 
				\item and $G_J(s)$ from the system
			\end{itemize}
		\end{itemize}
		\end{column}
		\begin{column}{0.4\textwidth}
			\begin{equation*}
				S(s) = \frac{1}{1+G_p(s)G_J(s)}
			\end{equation*}
			\begin{equation*}
				G_1(s) = \frac{G_{J}}{1+G_p(s)G_J(s)}
			\end{equation*}
		\end{column}
	\end{columns}
	\includegraphics{./pictures/sys.tikz}
\end{frame}
\begin{frame}
		\frametitle{Identify $G_p(s)$ and $G_J(s)$ in closed loop, Method 1}
	\begin{equation}
		\Delta \omega(s) = G_s(s)G_t(s)G_J(s)c(s)- G_J(s)\Delta P_{e}(s)
	\end{equation}
	\begin{itemize}
		\item Assume $G_c(s)$ to be known.
	\end{itemize}
	\begin{equation}
		G_p(s) = \frac{G_c(s)G_s(s)G_t(s)G_J(s)}{G_J(s)(1+\rho G_c(s)G_s(s))}
	\end{equation}
	\includegraphics{./pictures/sys_extended_Gp_GJ.tikz}
\end{frame}
\begin{frame}
		\frametitle{Identify $G_1(s)$ and $S(s)$ directly in closed loop, Method 2}
		\begin{equation}
			e(s) = G_1(s)\Delta P_{e}(s)+ S(s)r(s)
		\end{equation}
	\includegraphics{./pictures/sys_extended_S_G1.tikz}
\end{frame}
\begin{frame}
	\frametitle{Identifiability}
	\begin{itemize}
		\item<1-> The systems can be identified
		\item<2-> However, there is a lack of delay
		\item<2-> This is no problem if the effect of $v(s)$ in $\Delta P_e(s)$ is small compared to the effect of $v_l(s)$.
	\end{itemize}
		\includegraphics<1>{./pictures/sys_extended.tikz}
		\includegraphics<2->{./pictures/sys_extended_red.tikz}
\end{frame}
\begin{frame}
		\frametitle{Identifying $G_p(s)$ and $G_J(s)$ with different $v(s)$ amplitudes}
	\begin{columns}
		\begin{column}{0.3\textwidth}
			\begin{itemize}
				\item Mean frequency response from $1000$ simulations for each amplitude of $v(s)$
			\end{itemize}
		\end{column}
		\begin{column}{0.7\textwidth}
			\includegraphics{./pictures/sys_Gp_GJ.tikz}
		\end{column}
	\end{columns}
	\includegraphics[width=\textwidth]{./Validation/Plots/GP_GJ.tikz}
\end{frame}
\begin{frame}
		\frametitle{Identifying $S(s)$ with different $v(s)$ amplitudes}
	\begin{columns}
		\begin{column}{0.3\textwidth}
			\begin{itemize}
				\item Mean frequency response from $1000$ simulations for each amplitude of $v(s)$
			\end{itemize}
		\end{column}
		\begin{column}{0.7\textwidth}
			\includegraphics{./pictures/sys_S.tikz}
		\end{column}
	\end{columns}
	\includegraphics[width=\textwidth]{./Validation/Plots/S_1_vs_3.tikz}
\end{frame}
\begin{frame}
		\frametitle{Identifying $G_1(s)$ with different $v(s)$ amplitudes}
	\begin{columns}
		\begin{column}{0.3\textwidth}
			\begin{itemize}
				\item Mean frequency response from $1000$ simulations for each amplitude of $v(s)$
			\end{itemize}
		\end{column}
		\begin{column}{0.7\textwidth}
			\includegraphics{./pictures/sys_G1.tikz}
		\end{column}
	\end{columns}
	\includegraphics[width=\textwidth]{./Validation/Plots/G1_all.tikz}
\end{frame}
\begin{frame}
	\frametitle{Main contributions}
	\begin{itemize}
		\item Demonstrated two methods for finding transfer functions for checking the requirements in closed loop.
		\item The best method for finding the transfer functions is to first identify $G_p(s)$ and $G_J(s)$.
		\item Analytical validation of the demonstrated methods.
		\item Discussed the delay condition introduced earlier.
	\end{itemize}
\end{frame}


