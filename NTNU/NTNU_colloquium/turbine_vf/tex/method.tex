\section[Method]{Method for identifying governor transfer functions}
\begin{frame}
	\frametitle{Detailing of research questions}
	The method was designed such that the following questions could be answered regarding:
	\begin{enumerate}
		\item<1-> Is configuration easy?
		\item<2-> Are the results valid outside of the measurement time window?
		\item<3-> Can the results be obtained using a small measurement time window?
		\item<4-> Is the method fast?
	\end{enumerate}
\end{frame}
\begin{frame}
	\frametitle{Steps in the identification method}
	\begin{enumerate}
			\item<1-> Data collection
			\item<2-> Partitioning of data
			\item<3-> Preprocessing of data
			\item<4-> Vector fitting on all data partitions
			\item<5-> Cross validation and model selection
	\end{enumerate}
\end{frame}
\begin{frame}
	\frametitle{Data collection}
	\begin{itemize}
		\item<1-> Real PMU\footnote{Phasor measurement unit (PMU)} measurements from the Norwegian system
		\item<2-> Six generators at two different locations. 
		\item<3-> The data sets are one hour long and the sampling frequency was $50Hz$.
		\item<4-> Data sets with obvious nonlinearities such as ramping and saturation were discarded.
	\end{itemize}
\end{frame}
\begin{frame}
	\frametitle{Preprocessing of data}
	\begin{itemize}
		\item<1-> All data were decimated with a factor of 50.
		\item<2-> The average was removed from all datasets.
		\item<3-> The data was run through an antialiasing filter.
		\end{itemize}
\end{frame}
\begin{frame}
	\frametitle{Cross validation and model selection}
		\begin{itemize}
			\item<1-> No signal selection method was developed.
			\item<2-> Instead all possible cross validations were attempted.
			\item<3-> For a partition with length five minutes that adds up to $132$ possibilities for cross validation.
			\item<4-> Self validation was not allowed.
		\end{itemize}
\end{frame}
